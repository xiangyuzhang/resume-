%%%%%%%%%%%%%%%%%%%%%%%%%%%%%%%%%%%%%%%%%
% Medium Length Professional CV
% LaTeX Template
% Version 2.0 (8/5/13)
%
% This template has been downloaded from:
% http://www.LaTeXTemplates.com
%
% Original author:
% Trey Hunner (http://www.treyhunner.com/)
%
% Important note:
% This template requires the resume.cls file to be in the same directory as the
% .tex file. The resume.cls file provides the resume style used for structuring the
% document.
%
%%%%%%%%%%%%%%%%%%%%%%%%%%%%%%%%%%%%%%%%%

%----------------------------------------------------------------------------------------
%	PACKAGES AND OTHER DOCUMENT CONFIGURATIONS
%----------------------------------------------------------------------------------------

\documentclass{resume} % Use the custom resume.cls style

\usepackage[left=0.75in,top=0.3in,right=0.75in,bottom=0.4in]{geometry} % Document margins
\newcommand{\tab}[1]{\hspace{.2667\textwidth}\rlap{#1}}
\newcommand{\itab}[1]{\hspace{0em}\rlap{#1}}
\name{\large Xiangyu Zhang} % Your name
\address{2B E Plumtree Rd \\ Sunderland , MA. 01375} % Your address
%\address{123 Pleasant Lane \\ City, State 12345} % Your secondary addess (optional)
\address{321-305-9833 \\ xiangyuzhang@umass.edu} % Your phone number and email

\begin{document}

%----------------------------------------------------------------------------------------
%	EDUCATION SECTION
%----------------------------------------------------------------------------------------

\begin{rSection}{Education}

{\bf University of Massachusetts Amherst} \hfill {\em Expected: 12/2016} 
\\ M.S.in Electronic \& Computer Engineering \hfill { Overall GPA: 3.423}

{\bf Florida Institute of Technology} \hfill {\em 05/2014} 
\\ B.S.in Electronic \& Computer Engineering \hfill { Overall GPA: 3.5}
%Minor in Linguistics \smallskip \\
%Member of Eta Kappa Nu \\
%Member of Upsilon Pi Epsilon \\
\end{rSection}
%----------------------------------------------------------------------------------------
%	TECHNICAL STRENGTHS SECTION
%----------------------------------------------------------------------------------------

\begin{rSection}{Technical Strengths}

\begin{tabular}{ @{} >{\bfseries}l @{\hspace{3ex}} l }
Languages &  Verilog, C++, Python. \\
Tools & Quartus II, Linux, MINISAT, HSPICE, Cadence Virtuoso, Visual Studio, Git.\\
Courses & Computer Architecture, VLSI, Testing in VLSI, Verification, Algorithm, Computer Network.
\end{tabular}

\end{rSection}

%----------------------------------------------------------------------------------------
%	WORK EXPERIENCE SECTION
%----------------------------------------------------------------------------------------

\begin{rSection}{Research and Teaching Experience}\itemsep 2pt

\begin{rSubsection}{University of Massachusetts Amherst}{Research Assistant}{Area: Formal Equivalence Checking, SAT, Circuit Security}{06/2015 - Present}\itemsep -1pt
\item
	\begin{itemize}\itemsep -3pt
    
    	\item Oracle-Guided Incremental SAT Solver Development   
\begin{itemize}
		\item Participated in Oracle-Guided Incremental SAT Solver using \textbf{Python} and \textbf{C/C++}.\par
		\item Developed Circuit Equivalence Checking tool based on \textbf{MINISAT} using \textbf{Python}.\par 
\end{itemize}

	\item Camouflage Tools Development
\begin{itemize}
		\item Implemented Logic-level circuit obfuscation based on Anti-reverse Engineering using Transformable Interconnects in \textbf{Python}.\par
		\item Implemented Logic Anti-propagation in \textbf{Python}.\par		
\end{itemize} 

	\end{itemize}

\end{rSubsection}


%------------------------------------------------

\begin{rSubsection}{University of Massachusetts Amherst}{Teaching Assistant}{Course: ENG Computer Systems Lab I}{09/2015 - 01/2016}\itemsep -4pt
\item 
\begin{itemize}
\item Assisted in teaching design and analysis of digital systems using both hardware (Altera Complex Programmable Logic Device (CPLD) and \textbf{Verilog}) and software (Atmel \textbf{AVR}  ATmega32 microcontroller, assembly language and \textbf{C}).
\end{itemize}

\end{rSubsection}

\end{rSection}

%----------------------------------------------------------------------------------------
%	ACADEMIC PROJECTS
%----------------------------------------------------------------------------------------

\begin{rSection}{Academic Projects} 

\begin{itemize}\itemsep -2pt

\item Parallel Pattern Single Fault Simulation (PPSFP) based Fault Simulator \textit{(11/2015)}
    \begin{itemize}\itemsep -3pt
    \item Developed a parser and lexical analyzer based on \textbf{YACC \& LEX} to read ISCAS benchmark circuits and convert to customized double undirected graph using \textbf{C}.
    \item Implemented circuit levelization algorithm using \textbf{C/C++}.
    \item Implemented Parallel Pattern Single Fault Simulation (\textbf{PPSFP}) algorithm using \textbf{C/C++}.
    \end{itemize}

\item MIPS Simulating Model \textit{(10/2015)}
    \begin{itemize}\itemsep -3pt
    \item Simulated \textbf{MIPS} architecture in \textbf{C} and evaluated utilization of each gate.
    \end{itemize}

\item Universal Synchronous and Asynchronous Receiver Transmitter (\textbf{USART})  implementation on CPLD \textit{(11/2015)}
    \begin{itemize}\itemsep -3pt
    \item Designed a USART for MIDI device that will read a MIDI signal to interpret its content and display the note number in binary on seven LEDs using \textbf{CPLD} and \textbf{Verilog}.
    \end{itemize}

\item  General MIDI Explorer (GME) with Record/Playback implementation on AVR \textit{(12/2015)}
    \begin{itemize}\itemsep -3pt
    \item Developed record and playback function using USART and store function using EEPROM on ATmega32 \textbf{AVR} in \textbf{C}.
    \item Implemented Run-length encoding (RLE) to perform compression and uncompression in EEPROM.
    \end{itemize}
\end{itemize}

\end{rSection}
%----------------------------------------------------------------------------------------
%	PUBLICATION SECTION
%----------------------------------------------------------------------------------------

\begin{rSection}{Publications}\itemsep 2pt
\begin{itemize}
\item Duo Liu, \textbf{Xiangyu Zhang}, Cunxi Yu, Daniel Holcomb, Oracle-Guided Incremental SAT Solving to Reverse Engineer Camouflaged Logic Circuits, Design Automation and Test in Europe, \textbf{DATE 2016} (accepted).
\end{itemize}
\end{rSection}
%----------------------------------------------------------------------------------------

\end{document}
